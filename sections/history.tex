% história da aplicação
% mercado no passado e presente (do signal e concorrência)

\section{Contexto histórico}

\subsection{Origem}     % ver o titulo
Ao contrário de outras aplicações do mesmo estilo, como o \textit{WhatsApp}, que foram criadas sob a alçada de grandes empresas, com um grande financiamento desde o inicio, o \textit{Signal} foi criado como um projeto \textit{open-source} pelo investigador de cibersegurança \textbf{Moxie Marlinspike}. A primeira versão foi lançada em 2014. Desde o inicio que o propósito principal da aplicação é permitir aos seus utilizadores privacidade total quando comunicam usando o serviço, usando para isso um protocolo desenvolvido especialmente para o \textit{Signal}, o \textit{Signal Protocol}, que concede encriptação ponto a ponto ás comunicações feitas.

Apesar do \textit{Signal Protocol} ter sido criado para ser utilizado pelo \textit{Signal}, houve outras aplicações de outras empresas com interesse em usar este protocolo para permitir comunicações seguras sob a sua alçada:

\begin{itemize}
   \item \textbf{\textit{Facebook Messenger}} - Integrada em 2016 o protocolo como uma \textit{feature} adicional para possíveis comunicações mais seguras.
   
   \item \textbf{\textit{Skype}} - Integrada em 2018, como uma nova \textit{feature} em \textit{chats} do tipo \textit{Private Conversations}.
   
   \item \textbf{\textit{WhatsApp}} - Integrada em 2016, sendo que a utilização \textit{default} da aplicação utiliza o protocolo para todas as comunicações. 
\end{itemize}

Sendo que o \textit{Signal} pertence a uma organização sem fins lucrativos (atualmente, \textit{Signal Foundation}), não possui um plano financeiro estável, sobrevivendo de doações feitas por utilizadores e apoiantes da filosofia do serviço. Em 2018, o co-fundador do \textit{WhatsApp}, Brian Acton, doou 50 milhões de dólares à \textit{Signal Foundation} como uma forma de propulsionar o acesso facilitado à possibilidade de realizar comunicações seguras a qualquer cidadão. \cite{history}


\subsection{Presença no mercado}
